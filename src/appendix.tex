\chapter*{Anhang}
\chaptermarker{Anhang}
\addcontentsline{toc}{chapter}{Anhang}


\section{Repositories}\label{sec:repositories}
\begin{figure}[!ht]%
    \centering
    \subfloat[\centering fulibWorkflows\footnotemark]{{\includegraphics[width=6cm,height=6cm]{images/appendix/qr-fW} }}%
    \qquad
    \subfloat[\centering fulibWorkflows Web-Editor\footnotemark]{{\includegraphics[width=6cm,height=6cm]{images/appendix/qr-fWWE} }}%
    \caption{QR-Codes der für diese Arbeit erstellten Respositories}%
    \label{fig:repos}%
\end{figure}

\section{Aufnahme des Expertengespräches}\label{sec:aufnahme-des-expertengespräches}
\begin{figure}[!ht]
    \centering
    \includegraphics[width=6cm,height=6cm]{images/appendix/qr-rec}
    \caption{QR-Code zur Aufnahme des geführten Expertengespräches\footnotemark}
    \label{fig:rec}
\end{figure}

\footnotetext[1]{\url{https://github.com/fujaba/fulibWorkflows}}
\footnotetext[2]{\url{https://github.com/Kuenssi/fulibWorkflows-editor}}
\footnotetext[3]{\url{https://www.youtube.com/watch?v=2hYlh_xENDY}}

\newpage

\section{Workflow-Beispiel}\label{sec:workflow-beispiel}
\begin{listing}[!ht]
    \inputminted{yaml}{listings/appendix/pm.es.yaml}
    \caption{pm.es.yaml}
    \label{listing:pm-yaml}
\end{listing}
