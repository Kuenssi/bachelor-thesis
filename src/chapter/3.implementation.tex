\chapter{Implementierung}\label{ch:implementierung}
Hier bin ich mir tatsächlich unsicher was alles reinsoll.
Ich teile es auch erstmal in drei auf.

\section{fulibWorkflows}\label{sec:fulibworkflows2}
Was gibt es hier so interessantes zu sehen?
Gehen Sie weiter.

\subsection{fulibWorkflows Grammatik}\label{subsec:fulibworkflows-grammatik}
Beschreiben der ANTLR4 Grammatik für fulibWorkflows

Und natürlich auch wie mir das beim parsen der yaml geholfen hat.
Dat wird ne lange Sektion.

\subsection{Generierung dank fulibTools}\label{subsec:generierung-dank-fulibtools}
FulibTools gibt gute Anbindung an Graphviz, wenn man sowieso schon mit fulib arbeitet.

\subsection{schema}\label{subsec:schema}
Da hab ich lange dran gesessen.
Und ich glaube, selbst jetzt ist es kein gutes Schema.
Allerdings tut es was es soll.

\subsubsection{Mockups}
Eigenes Datenmodell gebaut.
Daraus die wichtigsten Infos gezogen.
Dank StringTemplates von antlr richtig easy zu bauen.
Gilt für Html als auch Fxml.

\section{editor-frontend}\label{sec:editor-frontend}
Alles was es zum FE so gibt.

\subsection{iframes}\label{subsec:iframes}
Naja der editor basiert einfach darauf, dass es iframes gibt.
Könnte man auch weg lassen?

\subsection{codemirror}\label{subsec:codemirror}
Eingerichtet und los ging es.
Noch einen eigenes kleines Hint Add on geschrieben.
Feddig.
Musste es erstmal simpel halten.
Gibt noch genügend zukünftige Ideen.

\subsection{angular split}\label{subsec:angular-split}
Danke an Adrian, der mit dazu geraten hat.

Nachdem ich mit purem css da grids erstellt habe, stand ich vor dem Problem der Veränderung von Größen.
Angular split löst das Problem auf eine wunderbare Art und Weise.

Die dreiteilung der View war damit wirklich einfach.
Auch wenn man aufpassen musste beim Verändern der größen, wenn man iframes benutzt.
Da musste ein kleiner Fix rein, der aber auch bereits von den machern vorgegeben war.

\section{editor-backend}\label{sec:editor-backend}
Das wird wieder kurz.

\subsection{Spring booterino}\label{subsec:spring-booterino}
Gabelstablinski hier drüben war dank ein paar Annotations schnell erstellt und macht bisher keine Probleme.

Die zip Datei wird im BE zusammengebaut.
Dafür gibt es schon alles fertig in java.util.zip.
Da hab ich nach gegoogelt und alles lief wie von selbst.

Na gut, ich muss zugeben, dass die zip vom BE ans FE schicken und dann richtig runterladen können,
ohne das mir alles um die Ohren fliegt schon einen Arbeitstag gebraucht hat.
Manchmal muss man lediglich seine Dummheit für einen Moment ablegen.