\section{editor-frontend}\label{sec:editor-frontend}
\todo{this}
Alles was es zum FE so gibt.

\subsection{iframes}\label{subsec:iframes}
\todo{this}
Naja der editor basiert einfach darauf, dass es iframes gibt.
Könnte man auch weg lassen?

\subsection{codemirror}\label{subsec:codemirror}
\todo{this}
Eingerichtet und los ging es.
Noch einen eigenes kleines Hint Add on geschrieben.
Feddig.
Musste es erstmal simpel halten.
Gibt noch genügend zukünftige Ideen.

\subsection{angular split}\label{subsec:angular-split}
\todo{this}
Danke an Adrian, der mit dazu geraten hat.

Nachdem ich mit purem css da grids erstellt habe, stand ich vor dem Problem der Veränderung von Größen.
Angular split löst das Problem auf eine wunderbare Art und Weise.

Die dreiteilung der View war damit wirklich einfach.
Auch wenn man aufpassen musste beim Verändern der größen, wenn man iframes benutzt.
Da musste ein kleiner Fix rein, der aber auch bereits von den machern vorgegeben war.
