\section{fulibWorkflows}\label{sec:fulibworkflows2}
\textit{fulibWorkflows} ist eine Java-Bibliothek, welche Workflow Beschreibungen als Eingabe bekommt, diese Eingabe
selbst parst und daraus HTML-/FXML-Mockups, Objektdiagramme und Klassendiagramme generiert.
Das Parsen wird über einen von Antlr generierten Parser übernommen, hierzu wird in Kapitel~\ref{subsec:fulibworkflows-grammatik} genauer auf
die zugrundeliegende Grammatik eingegangen.
Zudem wird auf die Limitationen des Parsers und der generierten Mockups eingegangen.

\subsection{fulibWorkflows Grammatik}\label{subsec:fulibworkflows-grammatik}
\todo{this}
Beschreiben der ANTLR4 Grammatik für fulibWorkflows

Und natürlich auch wie mir das beim parsen der yaml geholfen hat.
Dat wird ne lange Sektion.

\subsection{Generierung dank fulibTools}\label{subsec:generierung-dank-fulibtools}
\todo{this}
FulibTools gibt gute Anbindung an Graphviz, wenn man sowieso schon mit fulib arbeitet.

\subsection{schema}\label{subsec:schema}
\todo{this}
Da hab ich lange dran gesessen.
Und ich glaube, selbst jetzt ist es kein gutes Schema.
Allerdings tut es was es soll.

\subsubsection{Mockups}
\todo{this}
Eigenes Datenmodell gebaut.
Daraus die wichtigsten Infos gezogen.
Dank StringTemplates von antlr richtig easy zu bauen.
Gilt für Html als auch Fxml.
