\section{Backend des Web-Editors}\label{sec:editor-backend}
Das Backend des Web-Editors basiert auf einem mit Spring Initializr generiertem Java-Projekt.
Zusätzlich wurden bei der Konfiguration die Dependencies für eine Spring-Web-Anwendung hinzugefügt.
Neben den eben genannten Dependencies wurde lediglich fulibWorkflows als weitere Bibliothek zum Backend hinzugefügt.

Die verfügbaren Endpunkte des Backends werden in einem Controller bereitgestellt.
In diesem Fall ist dies der FulibWorkflowsController, welcher mit zwei Annotations versehen ist.
Die erste Annotation ist~\texttt{@Controller}, welche eine Klasse als Controller deklariert.
Bei der zweiten Annotation handelt es sich um~\texttt{@CrossOrigin()} mit welcher es ermöglicht wird, mit anderen Services zu interagieren.
Ohne diese Annotation kann der Mechanismus des \ac{CORS} nicht verwendet werden, wodurch Frontend und Backend
keine Anfragen vom jeweils anderem Service akzeptieren würden.
Im FulibWorkflowsController sind Endpunkte für die Generierung und den Download definiert.
Diese Definitionen werden in Listing~\ref{listing:endpoints} zur Veranschaulichung dargestellt.

\begin{listing}[!ht]
    \inputminted[firstnumber=15]{java}{listings/3.3/Endpoints.java}
    \caption{Definition der Endpunkte}
    \label{listing:endpoints}
\end{listing}

Beide Endpunkte können mit einem POST-Request angesprochen werden, da sowohl beim Download als auch der Generierung der Inhalt der
YAML-Datei vom Frontend mitgeschickt wird.
Dies sorgt dafür, dass bei beiden Endpunkten die Generierung durchgeführt wird, damit kein Abspeichern von Dateien stattfinden und keine IDs für
eine Beschreibung angelegt und verwaltet werden müssen.
Damit die Endpunkte auf diesen Inhalt zugreifen können, ist der erste Methodenparameter mit der \texttt{@RequestBody}-Annotation versehen.
Hierbei wird auf den Inhalt zugegriffen, welcher vom Frontend in der Post-Anfrage mitgesendet wurde.
Der Download-Endpunkt erhält zusätzlich zu der YAML-Beschreibung ebenfalls Query-Parameter.
Diese enthalten die aus dem Download-Pop-Up des Frontends angegeben Dateien, welche heruntergeladen werden sollen.
Mittels der \texttt{@ResponseBody}-Annotation können der Antwort des Backends weitere Daten hinzugefügt werden.
In welcher Form diese sind, resultiert aus dem Rückgabetyp der jeweiligen Methode unterhalb der Annotation.
Bei der generate-Methode wird ein JSON-Objekt als String zurückgegeben.
Die download-Methode wiederrum generiert ein Zip-Archiv und versendet die Daten als ByteArray.
Beide Methoden des Controllers reichen die erhaltenen Daten an den FulibWorkflowsService weiter, welcher sich um die Generierung über fulibWorkflows
und der Erstellung des Zip-Archives kümmert.

Im FulibWorkflowsService wird sowohl in der generate- als auch der createZip-Methode, welche vom Controller aufgerufen werden,
der YAML-String über fulibWorkflows generiert.
Dabei nutzt das Backend die Funktion des BoardGenerators von fulibWorkflows, um die generierten Dateien als ``Strings'' zurückzubekommen und somit weiterverarbeiten zu können.
Im Anschluss wird aus der Map von generierten Dateien ein \textit{GenerateResult} erstellt, welches von der generate-Methode zu einem JSON-String weiterverarbeitet und zurück an den Controller gegeben wird.
Nachdem das GenerateResult-Objekt in der createZip-Methode erstellt wurde, beginnt das Erstellen des Zip-Archivs.
Auf Grundlage der Query-Parameter werden die Dateien zum Zip-Archiv hinzugefügt, welche im Frontend ausgewählt werden.
In Abbildung~\ref{fig:export} wurden alle Dateien heruntergeladen, welche von fulibWorkflows generiert wurden.

\begin{figure}[h]
    \centering
    \includegraphics[width=0.6\textwidth]{images/3.3/export}
    \caption{Inhalt eines heruntergeladenen Zip-Archivs}
    \label{fig:export}
\end{figure}

Um die Dateien strukturiert zu halten, werden Diagramme und Mockups in eigenen Unterordnern abgelegt.
Hierbei wird sowohl zwischen Objektdiagrammen und dem Klassendiagramm, als auch zwischen HTML- und FXML-Mockups unterschieden.
Auf oberster Ebene wird die Workflow-Beschreibung und das generierte \ac{ES}-Board abgelegt, da diese
die Grundlage für alle weiteren Dateien bilden.
Das Klassendiagramm ist im Ordner \textit{class} abgelegt, die Objektdiagramme im \textit{diagrams}-Ordner.
Im Gegensatz hierzu werden die HTML-Mockups im \textit{mockups}-Ordner hinterlegt, die gleichen Mockups im FXML-Format werden
im \textit{fxmls}-Ordner platziert.
