\section{fulibWorkflows Web-Editor Backend}\label{sec:editor-backend}
Das Backend des Web-Editors basiert auf einem mit Spring Initializr generiertem Java Projekt.
Zusätzlich wurden bei der Konfiguration die Dependencies für eine Spring Web Anwendung hinzugefügt.
Neben den eben genannten Dependencies wurde lediglich fulibWorkflows als weitere Bibliothek zum Backend hinzugefügt.

Die verfügbaren Endpunkte des Backends werden in einem Controller bereitgestellt.
In diesem fall ist dies der FulibWorkflowsController, welcher mit zwei Annotations versehen ist.
Die erste Annotation ist~\texttt{@Controller}, welche der Anwendung mitteilt, dass diese Klasse als Controller agiert.
Bei der zweiten Annotation handelt es sich um \texttt{@CrossOrigin()} mit welcher es ermöglicht wird problemlos mit dem
Frontend zu interagieren.
Im FulibWorkflowsController sind Endpunkte für die Generierung und den Download definiert.
Diese Definitionen werden in Listing~\ref{listing:endpoints} zur Veranschaulichung dargestellt.

\begin{listing}[!ht]
    \inputminted[xleftmargin=20pt,linenos,firstnumber=15]{java}{listings/3.3/Endpoints.java}
    \caption{Definition der Endpunkte}
    \label{listing:endpoints}
\end{listing}

\todo{Warum beides Post?}

\todo{Was erhalten die Methoden, weitergeben an den service und dann was geben Sie zurück?}

\todo{Für Download -> Service erhält yaml vom Frontend und ruft fulibWorkflows auf, erstellt GenerateResult Objekt und
macht daraus JSON mit ObjectMapper das dann als string gespeichert wird}

\todo{Generierung von Zip Archiv -> Gleiches prozedere, außer umwandenln in json, anstelle davon erstellen von Zip Archiv}
