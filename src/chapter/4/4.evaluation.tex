\chapter{Evaluation}\label{ch:evaluation}
In diesem Kapitel wird mittels eines Expertengespräches eine Einschätzung über die Erfüllung der in
Kapitel~\ref{sec:ziele} definierten Ziele eingeholt.
Weiterhin sollte durch das Durchgehen der vorhandenen Beispiele aus der erstellten Anwendung Probleme und/oder weitere
Funktionen erkannt werden.

\section{Expertengespräch}\label{sec:expertengespraech}
Zuerst Fragen gestellt um Probleme beim RE herauszufinden und Adams Erfahrungen dazu zu sammeln.

Probleme bei UIs -> Differenz zwischen mockup und implementierung/nutzung -> Bezug auf die Motivation nehmen

Mit entsprechendem Tooling kostenredzuierung für unternehmen

Event Storming Fragen in der Praxis, entspricht das dem was vaughn vernon und brandolini sagten

UML in Event Storming nicht mehr enhalten, dennoch irgendwo sinnvoll?

Nutzung von Online Tool während Pandemie oder genereller Arbeit? -> Miro, bezug auf einleitung nehmen

Adam hat sich vorgestellt wie das Tool von ihm in der Praxis verwendet werden könnte, daraus kamen folgende Kommentare

- UML Datenmodellierung -> Editor nah an sowas wie dbdiagram.io was er schon nutzte für Datenmodellierung in der ersten Phase des RE

- Workflows -> Anordnung an einem Zeitstrahl damit genau das was beim ES auch gemacht wird -> Gut

- UI Probleme vom Anfang könnten durch die Anwendung gelöst werden -> Clickdummy der schnell generiert werden kann während ES session -> Gefällt ihm kann er sich gut vorstellen -> Ist aber mit Vorsicht zu genießen

\section{Auswertung}\label{sec:auswertung}
Während des Expertengespräches wurden Probleme besprochen, welche während dem \ac{RE} entstehen und welche Auswirkungen haben, um festzustellen,
ob die in dieser Arbeit präsentierte Anwendung alle oder einen Teil dieser Probleme lösen könnte.
Hierbei hat sich der Experte nach einer kurzen Präsentation der Anwendung vorgestellt diese in der Praxis zu nutzen und sieht dafür
Anwendungsbereiche, in erster Linie die Generierung der Mockups und der Verknüpfung, durch welches ein
Clickdummy\footnote{}
entsteht.\footnote{Anhang Expertengespräch ab Minute 42:25}
Dies bestätigt, dass die Anwendung einen positiven Beitrag zu einem Event Storming leisten kann.
Es wurden zudem weitere Funktionen besprochen, welche einen positiven Effekt auf die Verwendung haben können, diese werden in Kapitel~\ref{ch:ausblick} ausgeführt.
