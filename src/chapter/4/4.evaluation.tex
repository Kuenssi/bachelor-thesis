\chapter{Evaluation}\label{ch:evaluation}
In diesem Kapitel werden mittels eines Expertengespräches die bisher in dieser Arbeit gesammelten Erkenntnisse über Event Storming diskutiert.
Zudem wird die erstellte Web-Anwendung präsentiert und auf Nutzbarkeit überprüft.
Eine Aufnahme des gesamten Gespräches ist über den QR-Code im Anhang erreichbar.

Bevor dem Experten die in dieser Arbeit erstellte Anwendung präsentiert wurde, wurden Probleme beim~\ac{RE}, welche aus den Erfahrungen des Experten
hervorgingen, evaluiert.
Direkt zu Beginn des Gespräches bestätigte sich, dass es bei Gesprächen mit Auftraggebenden häufig zu einer Diskrepanz zwischen beschriebenen Benutzeroberflächen und der
tatsächlichen Implementierung kommt\footnote{Siehe~\ref{fig:rec} Expertengespräch ab Minute 1:30}.
Dabei fallen, wie in Kapitel~\ref{sec:motivation} beschrieben, Probleme bei der Benutzung einer Anwendung erst nach einer Implementierung auf, wodurch
weitere Anpassungen getätigt werden müssen.
Durch eben diese Anpassungen entstehen während der Softwareentwicklung weitere Kosten, da erneut Gespräche mit den Auftraggebenden darüber geführt werden müssen,
wie Oberflächen auszusehen haben, welche Funktionen noch fehlen oder falsch verstanden wurden.
Bei bisherigen \ac{ES}-Sessions wurden bisher keine Anwendungen zur Erstellung von Mockups verwendet, da der Fokus auf dem Ablauf und nicht den Oberflächen liegt.
Daraus resultiert, dass es mit einem entsprechendem Tooling ermöglicht werden kann eine Kostenreduzierung für das Unternehmen zu erreichen\footnote{Siehe~\ref{fig:rec} Expertengespräch ab Minute  3:43}.

Da als Methodik für diese Arbeit das \ac{ES} gewählt wurde, wurde ebenfalls darüber diskutiert, ob das~\ac{DDD} und~\ac{ES} auch in der praktischen
Anwendung während der Arbeit des Experten für ein produktiveres Arbeiten gesorgt haben.
Die Erkenntnisse und Beschreibung aus Kapitel~\ref{sec:event-storming} konnten vom Experten bestätigt werden\footnote{Siehe~\ref{fig:rec} Expertengespräch ab Minute  4:58}.
Dies bezieht sich sowohl auf Bestandteile des \ac{DDD} als auch die Verwendung von Event Storming für initiale Meetings.
Hierbei wurde vor allem auf das Verständnis des Prozesses während einer \ac{ES}-Session eingegangen.
Nicht nur Entwickelnde konnten dadurch besser verstehen, was für eine Anwendung benötigt wird, auch Teilnehmende aus dem Unternehmen erlangten
einen besseren Überblick über Bereiche in welchen diese nicht arbeiten.

Für \ac{ES}-Sessions wurden bereits Online-Anwendungen verwendet, dies geschah in erster Linie aufgrund der COVID-19-Pandemie\footnote{Siehe~\ref{fig:rec} Expertengespräch ab Minute  12:20}.
Dadurch wurde Miro, ein interaktives Online-Whiteboard, für Event Stormings verwendet.

Beim \ac{ES} werden bei einem Gespräch mit Auftraggebenden keine \ac{UML}-Diagramme zur Ermittlung eines Datenmodells verwendet,
da diese für nicht technisch versierte Personen keinen Mehrwert bieten.
Dennoch wird der generelle Einsatz von \ac{UML}-Diagrammen vor Entwicklungsstart einer Anwendung als sinnvoll erachtet\footnote{Siehe~\ref{fig:rec} Expertengespräch ab Minute  9:35}.

Nachdem grundlegende Probleme aus der Praxis und verwendete Technologien und Techniken während des ersten Teils des Gespräches festgelegt wurden,
folgte die Vorstellung der Anwendung mit allen dazugehörigen Funktionen.
Da eine der Funktionen der Anwendung das Erstellen von Datenmodellen mittels \ac{UML}-Diagrammen ist, verwies der Experte an den ähnlichen Aufbau der Anwendung
zu anderen Datenmodellierungs-Tools wie~\textit{dbdiagram.io}.
Dies kam beim Experten bereits in der Praxis zum Einsatz, um ein Datenmodell für Datenbanken zu erstellen.
Ebenso wie das in dieser Arbeit erstellte Tool, existiert dort ebenfalls ein Editor, in welchem mittels einer Beschreibungssprache ein Diagramm erzeugt wird,
welches ein Datenmodell darstellt.
Dies sei wie zuvor erwähnt besonders in der ersten Phase der Softwareentwicklung, dem~\ac{RE}, von Nutzen\footnote{Siehe~\ref{fig:rec} Expertengespräch ab Minute  29:29}.

Ebenfalls fiel auf, dass die Darstellung des \ac{ES}-Boards in mehreren Workflows einer Anordnung an einem Zeitstrahl gleicht.
Es wird somit ein zeitlicher Ablauf über Events modelliert.
Dies ist ein Grundbaustein des~\ac{ES}, somit geht dieser durch die Unterteilung in Workflows nicht verloren\footnote{Siehe~\ref{fig:rec} Expertengespräch ab Minute  39:17}.

Das zu Beginn beschriebene Problem von Diskrepanzen bei der Entwicklung von Oberflächen wurde durch die Möglichkeit der Generierung von
Mockups, welche grundlegende Funktionalität aufweisen, positiv aufgefasst.
Durch die Möglichkeit Mockups über Buttons untereinander zu verknüpfen, kann in der Anwendung ein Clickdummy erstellt werden.
Durch eine Generierung dieser Oberflächen während einer \ac{ES}-Session könne man den späteren Nutzenden, solange dieser innerhalb der Teilnehmenden ist,
ihren Arbeitsablauf anhand des Clickdummies durchführen lassen.
Daraus können bereits vor Start der Entwicklung Probleme und Ungereimtheiten bei der Modellierung gelöst werden\footnote{Siehe~\ref{fig:rec} Expertengespräch ab Minute  33:40}.

Es wurden zudem weitere Funktionen besprochen, welche einen positiven Effekt auf die Verwendung haben können, diese werden in Kapitel~\ref{ch:fazit} ausgeführt.
