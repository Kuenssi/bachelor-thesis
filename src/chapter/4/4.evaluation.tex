\chapter{Evaluation}\label{ch:evaluation}
In diesem Kapitel wird mittels eines Expertengespräches die bisher in dieser Arbeit gesammelten Erkenntnisse über Event Storming diskutiert.
Aus dieser Diskussion und einer anschließenden Präsentation des erstellten Web-Editors, wird evaluiert, ob die zu Beginn in Kapitel~\ref{sec:ziele}
gesetzten Ziele aus der Sicht des Experten erfüllt wurden.\newline
Eine Aufnahme des gesamten Gespräches ist über den QR-Code im Anhang erreichbar.

\section{Expertengespräch}\label{sec:expertengespraech}
Bevor dem Experten die in dieser Arbeit erstellte Anwendung präsentiert wurde, wurden Probleme beim~\ac{RE}, welche aus den Erfahrungen des Experten
hervorgingen, evaluiert.
Direkt zu Beginn des Gespräches bestätigte sich, dass es bei Gesprächen mit Kunden eine diskrepanz zwischen beschriebenen Benutzeroberflächen und der
tatsächlichen Implementierung gekommen ist.\footnote{Anhang Expertengespräch ab Minute 1:30}
Dabei fallen, wie in Kapitel~\ref{sec:motivation} beschrieben, Probleme bei der Benutzung einer Anwendung erst nach einer Implementierung auf, wodurch
weitere Anpassungen getätigt werden müssen.
Durch eben diese Anpassungen entstehen während der Softwareentwicklung weitere Kosten, da erneut Gespräche mit dem Kunden darüber geführt werden müssen,
wie Oberflächen auszusehen haben, welche Funktionen noch fehlen oder falsch verstanden wurden.
Daraus resultiert, dass es mit einem entsprechendem Tooling, egal ob eine Methodik des Requirements Engineerings oder eine Anwendung zur Mockup Generierung,
ermöglicht werden kann eine Kostenreduzierung für das Unternehmen zu erreichen.\footnote{Anhang Expertengespräch ab Minute 3:43}\newline
Da als Methodik für diese Arbeit das \ac{ES} gewählt wurde, wurde ebenfalls darüber diskutiert, ob das~\ac{DDD} und~\ac{ES}, auch in der praktischen
Anwendung während der Arbeit des Experten für ein produktiveres Arbeiten gesorgt hat.
Die Erkenntnisse und Beschreibung aus Kapitel~\ref{sec:event-storming} konnten vom Experten bestätigt werden.\footnote{Anhang Expertengespräch ab Minute 4:58}
Dies bezieht sich sowohl auf Bestandteile des \ac{DDD} als auch die Verwendung von Event Storming für initiale Meetings.
Hierbei wurde vor allem auf das Verständnis des Prozesses währen einer Event Storming Session eingegangen.
Nicht nur Entwickler konnten dadurch besser verstehen, was für eine Anwendung benötigt worden, auch Teilnehmer aus dem Unternehmen erlangten
einen besseren Überblick über Bereiche in welchen diese nicht arbeiten.\newline
Für Event Storming Session wurden bereits Online-Anwendungen verwendet, dies geschah in erster Linie aufgrund der Covid-Pandemie.\footnote{Anhang Expertengespräch ab Minute 12:20}
Dadurch wurde Miro\footnote{bereits in Kapitel~\ref{sec:existierende-konzepte} beschrieben}, ein interaktives Online-Whiteboard, für Event Stormings verwendet.\newline
Beim \ac{ES} werden bei einem Gespräch mit Kunden keine UML Diagramme zur Ermittlung eines Datenmodells verwendet,
da diese für nicht technisch versierte Personen keinen Mehrwert bieten.\footnote{Anhang Expertengespräch ab Minute 9:35}
Dennoch wird der generelle Einsatz von UML Diagrammen vor Entwicklungsstart einer Anwendung nicht als sinnlos erachtet,
da sich Entwickler im vorhinein ein Datenmodell überlegen können, um daraus weitere Fragen gegenüber den Kunden zu stellen.

Nachdem grundlegende Probleme aus der Praxis und verwendete Technologien und Techniken während dem ersten Teil des Gespräches festgelegt waren,
folgte die Vorstellung der Anwendung und allen dazugehörigen Funktionen.
Da eine der Funktionen der Anwendung, das Erstellen von Datenmodellen mittels UML Diagrammen ist, verwies der Experte an den ähnlichen Aufbau der Anwendung
zu anderen Datenmodellierungs-Tools wie~\textit{dbdiagram.io}.
Dies kam bereits in der Praxis zum Einsatz, um ein Datenmodell für Datenbanken zu erstellen.
Ebenso wie das in dieser Arbeit erstellte Tool, existiert dort ebenfalls ein Editor, in welchem mittels einer Beschreibungssprache ein Diagramm erzeugt wird,
welches ein grundlegendes Datenmodell darstellt.
Dies sei wie zuvor erwähnt besonders in der ersten Phase der Softwareentwicklung, dem~\ac{RE}, von Nutzen.\footnote{Anhang Expertengespräch ab Minute 29:29}\newline
Ebenfalls fiel auf, dass die Darstellung des Event Storming Boards in mehreren Workflows, ebenfalls einer Anordnung an einem Zeitstrahl gleicht.
Es wird somit ein zeitlicher Ablauf über Events modelliert, dies ist im \ac{ES} ebenfalls der Fall, wodurch die Unterteilung in Workflows nicht dazu führt,
dass ein Grundbaustein des \ac{ES} verloren geht.\footnote{Anhang Expertengespräch ab Minute 39:17}\newline
Das zu Beginn beschriebene Problem von Diskrepanzen bei der Entwicklung von Oberflächen wurde durch die Möglichkeit der Generierung von
Mockups, welche grundlegende Funktionalität aufweisen, positiv aufgefasst.
Durch die Möglichkeit Mockups über Buttons untereinander zu verknüpfen, kann in der Anwendung ein Klickdummy erstellt werden.
Bei einem Klickdummy handelt es sich um Oberflächen, welche so miteinander verbunden sind, sodass diese eine Anwendung emulieren.
Durch eine Generierung dieser während einer Event Storming Session könne man den späteren Nutzer, solange diese innerhalb der Teilnehmer sind,
ihren Arbeitsablauf anhand des Klickdummies durchführen lassen.
Daraus kann bereits vor Start der Entwicklung auf Probleme und Ungereimtheiten bei der Modellierung gelöst werden.\footnote{Anhang Expertengespräch ab Minute 33:40}\newline
Es wurden zudem weitere Funktionen besprochen, welche einen positiven Effekt auf die Verwendung haben können, diese werden in Kapitel~\ref{ch:ausblick} ausgeführt.
