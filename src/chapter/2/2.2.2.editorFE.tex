\subsection{fulibWorkflows Web-Editor FE}\label{subsec:fulibworkflows-web-editor}
\todo
Da brauchte es etwas mehr als beim BE\@.
Die Entscheidungen für die verwendeten Technologien im Frontend wurden aufgrund
der Idee der Integrierung vom Editor auf fulib.org getroffen.

\subsubsection{Angular}
\todo
Wir kennen es.
Wir lieben es.

\subsubsection{Bootstrap}
\todo
Alles für den Dackel, alles für den Club unser Leben für die schön gestylten FEs.
Simpel, oder?
Ja, okay.
Man nutzt dann auch ng-bootstrap für Angular Anwendungen.
Natürlich auch noch bootstrap-icons.
Will ja niemand traurig machen und von Adrian verdroschen werden.

\subsubsection{Codemirror}
\todo
Schönes Ding.
Ngx-codemirror ist es dann speziell für eine Angular Anwendung geworden.
Eigentlich alles out of the box benutzt.

\subsubsection{Angular-split}
\todo
Find ich schon gut zu erwähnen.
Ohne die geile dependency wäre das FE nie so pornös geworden.

\subsubsection{file-saver}
\todo
Weitere erwähnenswerte dependency.
Wird genutzt um Dateien, die vom BE kommen, auch herunterladen zu können.

\subsubsection{ajv}
\todo

\subsubsection{js-yaml}
\todo
