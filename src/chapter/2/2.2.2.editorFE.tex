\subsection{fulibWorkflows Web-Editor Frontend}\label{subsec:fulibworkflows-web-editor}
Der Web-Editor für fulibWorkflows besteht aus einem Frontend und einem Backend.
Dieses Kapitel beschäftigt sich mit den Technologien, welche für das Frontend verwendet wurden.
Hierbei wurde die Entscheidung über die verwendeten Technologien für die Integrierung des Editors in die Webseite \url{https://fulib.org/editor}.
Über eine Integrierung wird in Kapitel~\ref{ch:ausblick} näher eingegangen.

\subsubsection{Angular}
Die Grundlage für das Frontend ist Angular.
Angular ist ein Framework für das Designen von Applikationen und gleichzeitig eine Entwicklungsplattform.\cite*{angular}
Für Entwickler ist das Angular Command-Line-Interface ein wichtiger Bestandteil bei der Entwicklung mit Angular.
Neben der Generierung einer neuen Anwendung, können ebenfalls neue Komponenten, Services und Module generiert werden.
Die \textit{Komponenten} dienen der Strukturierung einer Anwendung und enthalten verschiedene Abschnitte einer Anwendung.
Hierbei besteht ein großer Vorteil darin, Komponenten modular zu gestalten.
Dies bedeutet, dass man Komponenten so entwickelt, dass diese in der Anwendung wiederverwendet werden können, sollte dies möglich sein.
Eine Komponente besteht aus drei einzelnen Bereichen:

\begin{enumerate}
    \item Der Logik/dem Code, welche/r in Typescript verfasst wird.
    \item Einem Template, welches eine HTML-Datei ist.
    \item Styles, welche im Template eingebunden werden können, um die Oberfläche grafisch zu verändern.
\end{enumerate}

Im Template werden neben den Styles auch Bestandteile des Code-Segments verwendet um Daten dynamisch anzeigen zu können.
Ein neu generiertes Angular Projekt bietet allerdings nur die Grundlage einer Anwendung.
Um weitere Funktionalitäten in der Anwendung verwenden zu können, können sogenannte Package Manager, wie \textit{npm} oder \textit{yarn}
verwendet werden, um neue Bibliotheken einzubinden.

Dies soll für dieses Kapitel genügen, da Angular in Gänze zu erklären den Rahmen dieser Arbeit überschreiten würde.
In Kapitel~\ref{sec:editor-frontend} wird auf weitere Funktionen genauer eingegangen und dies anhand der Implementierung erklärt.
Folgend werden eben erwähnte Bibliotheken erläutert, welche die wichtigsten Bestandteile der Anwendung sind.

\subsubsection{Bootstrap}
Um eine ansehnlichere Oberfläche zu gestalten, welche einheitlich mit der von \textit{fulib.org} sein soll, wurde Bootstrap zum
Stylen der OberflächenElemente verwendet.
Bootstrap bietet diverse Komponenten, wie Eingabefelder, Buttons, Menüs, Pop-Ups und viele weitere.
Neben den Styles enthalten diese auch zusätzliche Funktionen, welche im Kontext der Komponente sinnvoll sind.
Dabei bleibt es weiterhin abänderbar, um dem Entwickler mehr Freiheiten zur Gestaltung einer Oberfläche zu geben.
Weiterhin ermöglicht es Bootstrap das Layout, also die Anordnung, von Komponenten auf einer Seite zu sortieren.
Dies beginnt bei der Größe einer Komponente bis hin zu der Anordnung in der Horizontale und Vertikale.\cite*{bs}

Zusätzlich zu Bootstrap ist es möglich Bootstrap Icons zu verwenden.
Hierbei handelt es sich über 1500 Icons, welche im open source Prinzip zugänglich sind.\cite*{bsIcons}

\subsubsection{Codemirror}
\todo
Schönes Ding.
Ngx-codemirror ist es dann speziell für eine Angular Anwendung geworden.
Eigentlich alles out of the box benutzt.

\subsubsection{Angular-split}
\todo
Find ich schon gut zu erwähnen.
Ohne die geile dependency wäre das FE nie so pornös geworden.

\subsubsection{file-saver}
\todo
Weitere erwähnenswerte dependency.
Wird genutzt um Dateien, die vom BE kommen, auch herunterladen zu können.

\subsubsection{ajv}
\todo

\subsubsection{js-yaml}
\todo
