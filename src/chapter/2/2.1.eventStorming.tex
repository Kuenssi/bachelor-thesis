\section{Event Storming}\label{sec:event-storming}
In diesem Unterkapitel werden die grundlegenden Prinzipien und Ziele des \ac{DDD} erläutert, welche von Vaughn Vernon in seinem Buch
\textit{Domain-Driven Design Distilled} definiert hat.\cite*{dddd}
Nachdem diese Grundlage vorhanden ist, wird darauf aufbauend erklärt, welche Symbiose aus dem \ac{DDD} und dem \ac{ES} entsteht und wie
dies zu einer Softwareentwicklung beiträgt.
Zur Veranschaulichung eines Event Stormings wird ein von Alberto Brandolini beschriebener \ac{ES}-Workshop genutzt.
Abschließend werden die Änderungen und Erweiterungen, welche im Kontext dieser Arbeit vorgenommen wurden, erklärt.

\subsection{Domain-Driven Design}\label{subsec:domain-driven-design}
Das \ac*{DDD} ist nicht nur für die erste Phase der Softwareentwicklung praktisch, sondern ebenso für das Umstrukturieren bestehender Projekte.
Ein grundlegendes Ziel des \ac{DDD} ist es ein Projekt in sogenannte \textit{Bounded Contexts} zu unterteilen und damit zu umgehen, dass
die Anwendung aus einem riesigen aufgeblähten Modell besteht.
Um dieses Ziel zu erreichen, ist es wichtig in Gesprächen mit Domänenexperten die wichtigsten Punkte eines \textit{Bounded Context} zu evaluieren.
Domänenexperten können in jedem Bereich eines Unternehmens gefunden werden.
Es ist nötig ein möglichst breites Spektrum an Personen zu haben, um den gesamten zu entwickelnden Prozess zu verstehen und für die Entwickler verständlich zu machen.
Dabei ist es wichtig, dass alle Personen, welche am Prozess des \ac{DDD} teilnehmen eine einheitliche Sprache zu entwickeln.
Diese einheitliche Sprache beschreibt Vernon als~\textit{Ubiquitous Language}.\footnote{Seite 7 in~\cite*{dddd}}
Eine allgegenwärtige Sprache (\textit{Ubiquitous Language}) zu entwickeln, ist ein fortlaufender Prozess.
Initial ist es wichtig, dass zwischen den verschiedenen Domänenexperten und den Entwicklern diese einheitliche Sprache entsteht, welche
nicht nur das Verständnis zwischen den beiden Parteien, sondern auch mit in das Modell einfließen soll.
Vernon selbst nennt Event Storming, als eine Möglichkeit um eine~\textit{Ubiquitous Language} zu entwickeln.\footnote{Seite 112 folgende in~\cite*{dddd}}
Wie zuvor bereits erwähnt ungeachtet, ob ein Projekt neu oder bereits seit zwei Jahren in Entwicklung ist.

\subsection{Event Storming}\label{subsec:allgemein}
\begin{itemize}
    \item Alberto Brandolini und das ES
    \item Wie verläuft so ein ES Workshop (Beschrieben in seinem Buch, mehrfach)
    \item Wichtigsten Eckpunkte
    \item Warum ist es besser als Brain Storming oder ähnliches?
    \item Beispielhaftes Event Storming Board (Bild und beschreibungstext um später darauf bezug nehmen zu können)
\end{itemize}

\subsection{Erweiterung}\label{subsec:erweiterung}
\todo{Hier das vorherige Kapitel abwarten um alle Änderungen/Erweiterungen besser daran fest zu machen.}
\begin{itemize}
    \item Erweiterungen für Wirtschaft (Pages -> daraus generierte Mockups, abgehen von dem "Wir wollen keinen PC benutzen" des ES)
    \item Ideen für die Lehre (Wird in dieser Arbeit nicht näher beleuchtet, da es für den Beleg der Funktionalität nicht mehr möglich ist dies ausreichend in der Bearbeitungszeit zu machen)
    \item ES -> Ablauf von Schritten -> Albert -> Workflow (Arbeitsablauf) beschreibungen -> Mögliche Idee zum besseren Nahebringen von komplexeren Abläufen in Vorlesungen. (Verbildlichung)
\end{itemize}
