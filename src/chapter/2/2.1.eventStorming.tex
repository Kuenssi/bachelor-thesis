\section{Event Storming}\label{sec:event-storming}
Dieses Unterkapitel befasst sich mit der Herkunft des \textit{Event Stormings}, dem \ac*{DDD}.
Zudem wird ein klassischer Ablauf eines \textit{Event Stormings} erläutert und daran die Vorteile dieser Methodik für das Requirements Engineering beleuchtet.
Abschließend werden die Änderungen und Erweiterungen, welche im Kontext dieser Arbeit vorgenommen wurden, erklärt.

\subsection{Allgemein}\label{subsec:allgemein}
\todo{Bevor ich dieses und das folgende Unterkapitel schreibe, erst noch mal das~\cite*{dddd} und~\cite*{introES} lesen.}
\begin{itemize}
    \item Alberto Brandolini und das ES
    \item DDD als Grundlage
    \item Wie verläuft so ein ES Workshop (Beschrieben in seinem Buch, mehrfach)
    \item Wichtigsten Eckpunkte
    \item Warum ist es besser als Brain Storming oder ähnliches?
    \item Beispielhaftes Event Storming Board (Bild und beschreibungstext um später darauf bezug nehmen zu können)
\end{itemize}

\subsection{Erweiterung}\label{subsec:erweiterung}
\todo{Hier das vorherige Kapitel abwarten um alle Änderungen/Erweiterungen besser daran fest zu machen.}
\begin{itemize}
    \item Erweiterungen für Wirtschaft (Pages -> daraus generierte Mockups, abgehen von dem "Wir wollen keinen PC benutzen" des ES)
    \item Ideen für die Lehre (Wird in dieser Arbeit nicht näher beleuchtet, da es für den Beleg der Funktionalität nicht mehr möglich ist dies ausreichend in der Bearbeitungszeit zu machen)
    \item ES -> Ablauf von Schritten -> Albert -> Workflow (Arbeitsablauf) beschreibungen -> Mögliche Idee zum besseren Nahebringen von komplexeren Abläufen in Vorlesungen. (Verbildlichung)
\end{itemize}
