\section{Event Storming}\label{sec:event-storming}
In diesem Unterkapitel werden die grundlegenden Prinzipien und Ziele des \ac{DDD} erläutert, welche von Vaughn Vernon in seinem Buch
\textit{Domain-Driven Design Distilled} definiert hat.\cite*{dddd}
Nachdem diese Grundlage vorhanden ist, wird darauf aufbauend erklärt, welche Symbiose aus dem \ac{DDD} und dem \ac{ES} entsteht und wie
dies zu einer Softwareentwicklung beiträgt.
Zur Veranschaulichung eines Event Stormings wird ein von Alberto Brandolini beschriebener \ac{ES}-Workshop genutzt.
Abschließend werden die Änderungen und Erweiterungen, welche im Kontext dieser Arbeit vorgenommen wurden, erklärt.

\subsection{Domain-Driven Design}\label{subsec:domain-driven-design}
\todo{Was sind Bounded Contexts?}

\todo{Wer sind Domänenexperten?}

\todo{Warum ist Ubiquitous Language so wichtig?}

\todo{Wie wird die Erstellt im DDD? -> ES als schöne Möglichkeit}

\subsection{Event Storming}\label{subsec:allgemein}
\begin{itemize}
    \item Alberto Brandolini und das ES
    \item Wie verläuft so ein ES Workshop (Beschrieben in seinem Buch, mehrfach)
    \item Wichtigsten Eckpunkte
    \item Warum ist es besser als Brain Storming oder ähnliches?
    \item Beispielhaftes Event Storming Board (Bild und beschreibungstext um später darauf bezug nehmen zu können)
\end{itemize}

\subsection{Erweiterung}\label{subsec:erweiterung}
\todo{Hier das vorherige Kapitel abwarten um alle Änderungen/Erweiterungen besser daran fest zu machen.}
\begin{itemize}
    \item Erweiterungen für Wirtschaft (Pages -> daraus generierte Mockups, abgehen von dem "Wir wollen keinen PC benutzen" des ES)
    \item Ideen für die Lehre (Wird in dieser Arbeit nicht näher beleuchtet, da es für den Beleg der Funktionalität nicht mehr möglich ist dies ausreichend in der Bearbeitungszeit zu machen)
    \item ES -> Ablauf von Schritten -> Albert -> Workflow (Arbeitsablauf) beschreibungen -> Mögliche Idee zum besseren Nahebringen von komplexeren Abläufen in Vorlesungen. (Verbildlichung)
\end{itemize}
