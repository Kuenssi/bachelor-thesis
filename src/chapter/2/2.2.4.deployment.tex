\subsection{Deployment}\label{subsec:deployment}
\todo
Ein Web Editor will natürlich für alle erreichbar sein.
Und fulibWorkflows muss auch irgendwo bereitgestellt werden, damit es das Backend und alle anderen
interessierten benutzen können.

\subsubsection{MavenCentral}\label{subsubsec:mavencentral}
\todo
MavenCentral ein wirklicher Hussarones was das publishen angeht.
Glücklicherweise ist fulibWorkflows Teil der Fujaba Tool Suite, wodurch die benötigten
Zugriffsrechte bereits vorhanden und andere Libraries bereits gepublished wurden.
Hierdurch war es recht schnell möglich mit dem zuvor erworbenem Wissen fulibWorkflows
zu publishen.

\subsubsection{Heroku}\label{subsubsec:heroku}
\todo
Der Web-Editor soll immer erreichbar sein.
Dies ist durch Heroku nur bedingt möglich.
Heroku bietet allerlei Möglichkeiten verschiedenste Anwendungen bereitzustellen.
Auch mit einem kostenlosen Plan ist es ohne Probleme möglich solch kleine Anwendungen bereitzustellen.

FE Deployment war easy, auch wenn ich erstmal wieder in eins meiner früheren Projekte gucken musste.
BE Deployment war kniffliger, doch man ist nie der erste der eine Spring Boot application
auf Heroku deployen will.
Daher Tutorial reingefahren und ab ging der gebutterte Lachs.
