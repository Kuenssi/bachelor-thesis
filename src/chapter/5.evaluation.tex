\chapter{Evaluation}\label{ch:evaluation}
Für die Anwendungen in verschiedenen Bereichen werden diese zunächst festgelegt und beschrieben.
Anschließend wurden User Evaluationen durchgeführt.
Sowohl Lehre, mit Studis aus den aktuellen Veranstaltungen, als auch der Wirtschaft, Adam Malik incoming.

\section{Anwendungszwecke}\label{sec:anwendungszwecke}
Das Tool wird für mehrere Bereiche entworfen.
Daher hier mal die Anwendungszwecke die Albert und ich uns ausgedacht haben.

\subsection{Wirtschft - RE}\label{subsec:wirtschft---re}
Event Storming nachvollziehbar für alle machen.
Htmls einfach noch mal öffnen.

Besser mit Kunden über das Layout sprechen können -> Mockups mit grundlegendem Styling
Kunden können einen Workflow mal durchklicken -> Next prev page

\subsection{Lehre - FG}\label{subsec:lehre---fg}
Albert nutzte die Event Storming Boards auch im WS21/22 bereits in PM und MS.
Veranschaulichung von Abläufen.

\subsubsection{PM}
Erstellen von Objekt, Klassendiagrammen über Workflows.

\subsubsection{MS}
Verstehen von Abläufen in komplexen Systemen.
Kommunikation zwischen Services

\section{User-Tests}\label{sec:user-tests}
Leuten wurden das Tool, die Doku und eine Aufgabe gegeben.
Ich stand für Fragen bereit.

\subsection{Adam Malik}\label{subsec:adam-malik}
Gute Frage, was ich mit Adam mache.

\subsection{PM-Studis}\label{subsec:pm-studis}
Was sie tun sollen:
Ablauf eines Mühle spiels in simplen schritten aufschreiben
Schauen, ob das objektdiagrammen/klassendiagramm aussieht wie in der Veranstaltung

\subsection{MS-Studis}\label{subsec:ms-studis}
Was sie tun sollen:
Registration, Login, Order

Die drei workflows erstellen und dann schauen, ob die Parallelen zu Alberts Beispiel und der
Shop-Warehouse Nummer erkennbar ist.
