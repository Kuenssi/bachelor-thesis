\chapter{Einleitung}\label{ch:einleitung}
\todo{this}

\section{Motivation}\label{sec:motivation}
\todo{this}
Ach ja das Requirements Engineering in der agilen Softwareentwicklung ist
und bleibt ein leidiges Thema.
Dafür wurde von Alberto Brandolini das Event Storming ins Leben gerufen.
Namensvetter Albert dachte sich: ''Ja, geil - Hab ich Bock drauf.''
So begann er es zu erweitern und nun sind wir alle hier und schauen uns das an.

\section{Ziele}\label{sec:ziele}
\todo{this}
Noch kurz warum Event Storming eine gute Idee ist und hilfreich in der
Anforderungsanalyse sein kann.
Natürlich schon mal kurz ansprechen, dass für Albert(o)s Event Storming eine
Beschreibungssprache entwickelt wurde und man diese zum einfachen Bedienen
mit einem Web-Editor versehen hat.

\section{Methodik}\label{sec:methodik}
\todo{this}
Expertengespräch zum Prüfen der gesetzten Ziele.
Was ist das und warum ist es für die Ziele ausreichend?

\section{Existierende Konzepte}\label{sec:existierende-konzepte}
\todo{this}
Oldschool alles auf papier -> keine mockups direkt mit framework
Web Editoren wie Miro für das Erstellen von Boards.

\section{Aufbau der Arbeit}\label{sec:aufbau-der-arbeit}
\todo{this}
Zuerst grundlegende Ideen und Technologien.
Anschließend die Implementierung beschreiben.
Evaluation mittels Expertengespräch mit Adam Malik.
Fazit bezüglich Ziele und outcome.
Ausblick für das Tool.
