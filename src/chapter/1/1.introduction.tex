\chapter{Einleitung}\label{ch:einleitung}
\todo{this}

\section{Motivation}\label{sec:motivation}
\todo{this}
Ach ja das Requirements Engineering in der agilen Softwareentwicklung ist
und bleibt ein leidiges Thema.
Dafür wurde von Alberto Brandolini das Event Storming ins Leben gerufen.
Namensvetter Albert dachte sich: ''Ja, geil - Hab ich Bock drauf.''
So begann er es zu erweitern und nun sind wir alle hier und schauen uns das an.

\section{Ziele}\label{sec:ziele}
Aus der Motivation laassen sich mehrere Ziele extrahieren:

\begin{itemize}
    \item Entwicklung einer Beschreibungssprache für Event Storming
    \item Verarbeitung der Beschreibungssprache in benutzerfreundliche Formate
    \item Entwicklung einer Web-Anwendung zur Erstellung einer \ac{ES} Beschreibung
    \item Darstellung der verarbeiteten Beschreibung in der Web-Anwendung
\end{itemize}

Das Ergebnis dieser Arbeit soll alle vorherigen Ziele erfüllen, somit soll eine Web-Anwendung entwickelt werden,
welche den Ablauf eines Event Storming Workshops unterstützt.
Zusätzlich soll auch nach einem solchen Workshop die Möglichkeit vorhanden sein auf diese Daten zugreifen und Aktualisierungen
im Verlauf der Entwicklung vornehmen zu können.

\section{Existierende Konzepte}\label{sec:existierende-konzepte}
\ac{ES}, wie es klassisch von Alberto Brandolini entwickelt wurde, basiert auf der Arbeit in einem Raum mit vielen Post-Its.
Dadurch kann jede Person, welche am \ac{ES} teilnimmt, frei und jederzeit neue Events erstellen und anbringen kann.
Sollte ein Zusammenkommen in einem Raum nicht möglich sein, sei es durch internationale Teilnehmer oder, wie es momentan der Fall ist,
durch eine Pandemie, welche aufgrund von Sicherheitsmaßnahmen kein Treffen mit größeren Gruppen ermöglicht, können Online-Tools genutzt werden.
Beispiele für interaktive Online Boards sind Miro\footnote{https://miro.com/}, Conceptboard\footnote{https://conceptboard.com/} und MURAL\footnote{https://www.mural.co/}.
Hierbei kann jeder Teilnehmer ebenfalls selbstständig neue Events erstellen und auf dem Board platzieren.
Diese Aktualisierungen werden für alle Teilnehmer synchronisiert.

\section{Aufbau der Arbeit}\label{sec:aufbau-der-arbeit}
In Kapitel~\ref{ch:grundlagen} werden zuerst die Grundlagen des~\ac{ES} und die verwendeten Technologien
zur Erstellung der zuvor erwähnte Anwendungen erläutert.
Hierbei sind die Technologien entsprechend der zugehörigen Anwendung zugeordnet.
Anschließend befasst sich Kapitel~\ref{ch:implementierung} mit der Implementierung der Anwendungen.
Dabei wird genauer auf die Anwendung der aus Kapitel~\ref{ch:grundlagen} erläuterten Technologien eingegangen.
Um den Nutzen der erstellten Anwendungen zu prüfen, wird in Kapitel~\ref{ch:evaluation} eine Evaluierung mittels Expertengespräch durchgeführt.
Die zuvor aufgestellten Ziele sollen somit, gemeinsam mit einem Experten, als erfüllt oder nicht erfüllt evaluiert werden.
Im folgenden Kapitel wird, unter Einbezug der in der Evaluation erlangten Kenntnisse, ein Fazit zu den gestellten Fragen und Zielen erstellt.
Zusätzlich soll die Umsetzung der zuvor festgelegten Funktionen anhand der Implementierung überprüft werden.
Zuletzt werden in Kapitel~\ref{ch:ausblick} ein Ausblick zur weiteren Entwicklung der Anwendung gegeben.
Unter anderem wird auch der mögliche Nutzen in der Lehre angesprochen.
