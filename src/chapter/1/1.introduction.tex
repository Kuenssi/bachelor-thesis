\chapter{Einleitung}\label{ch:einleitung}

\begin{quote}
    \textit{Zwei kleine Änderungswünsche (jetzt, wo ich es sehe) ...}
\end{quote}


Variationen dieses Satzes\footnote{, welcher von einer auftraggebenden Person aus einem realen Projekt des Fachgebietes stammt}
begegnen Softwareentwickelnden bei der agilen Softwareentwicklung für eine personalisierte Anwendung häufiger.
In der agilen Softwareentwicklung können sich Anforderungen an die Anwendung stetig verändern.
Neue Features werden von den Auftraggebenden ausprobiert, wodurch schnelles Feedback entsteht.
Dies verbessert zwar das Endprodukt, allerdings sorgt diese Art der Entwicklung für einen Mehraufwand aufseiten der Entwickelnden.
In diesem Prozess ist die technische Abschätzung der Entwickelnden unerlässlich.
Allerdings stellt dies auch eine Herausforderung für Softwareentwickelnde, im Folgenden ``Entwickelnde'', dar, sollten diese keine Erfahrung im
\ac{RE} haben.
In Gesprächen mit den Auftraggebenden die korrekten Anforderungen zu verstehen und umzusetzen stellt ohne eine passende Methodik zum
\ac{RE} ein Problem dar.

\section{Motivation}\label{sec:motivation}
Um komplexe Arbeitsabläufe zu Beginn einer Softwareentwicklung zu verstehen, existiert die Methode des Event Stormings von Alberto Brandolini.
Hierbei setzen sich Entwickelnde und Sachverständige zusammen und erarbeiten in einem Workshop die in der Software darzustellenden Arbeitsabläufe.
Sobald eine Anwendung Benutzeroberflächen enthält, benötigt sie ebenso Mockups, wodurch die Entwickelnden Vorschläge für eine Benutzeroberfläche
darlegen können, um ein schnelles Feedback der Auftraggebenden zu erlangen.
Diese beiden Methodiken können die soeben beschriebenen Probleme lösen.
Doch während dem Event Storming können bereits Aussagen über eine mögliche Benutzeroberfläche getroffen werden.
Dies wird im Expertengespräch in Kapitel~\ref{ch:evaluation} bestätigt.
Eine Kombination aus den beiden Methodiken könnte somit in einem größeren Nutzen resultieren.
Aus dieser Idee entstand das Thema dieser Arbeit und der Versuch das Event Storming sinnvoll zu erweitern beziehungsweise den Prozess zu unterstützen.

\section{Ziele}\label{sec:ziele}
Aus der Motivation lassen sich mehrere Ziele extrahieren:

\begin{itemize}
    \item Entwicklung einer Beschreibungssprache für \ac{ES}
    \item Verarbeitung der Beschreibungssprache in benutzerfreundliche Formate
    \item Entwicklung einer Web-Anwendung zur Erstellung einer \ac{ES}-Beschreibung
    \item Darstellung der verarbeiteten Beschreibung in der Web-Anwendung
\end{itemize}

Das Ergebnis dieser Arbeit soll alle extrahierten Ziele erfüllen.
Daraus soll eine Web-Anwendung entwickelt werden, welche den Ablauf eines \ac{ES}-Workshops unterstützt.
Zusätzlich soll nach einem solchen Workshop die Möglichkeit vorhanden sein auf diese Daten zuzugreifen, um Aktualisierungen
im Verlauf der Entwicklung vornehmen zu können.


\section{Existierende Konzepte}\label{sec:existierende-konzepte}
Event Storming, wie es klassisch von Alberto Brandolini entwickelt wurde, basiert auf der Arbeit in einem Raum mit vielen Post-its\textsuperscript{\textregistered}.
Dadurch kann jede Person, welche am \ac{ES} teilnimmt, frei und jederzeit neue Events erstellen und anbringen.
Das Zusammenkommen in einem Raum ist nicht immer möglich.
Ein Grund können Teilnehmende aus verschiedenen Ländern sein.
In der aktuelle herrschenden Pandemie konnten aufgrund von Sicherheitsmaßnahmen keine größeren Gruppen in einem Raum versammelt werden.
Dadurch ist der Einsatz von Online-Tools eine Möglichkeit diese Probleme zu umgehen.
Ein Beispiel für ein interaktives Online-Board ist Miro.\footnote{https://miro.com/}
Hierbei kann jeder Teilnehmende ebenfalls selbstständig neue Events erstellen und auf dem Board platzieren.
Zudem können Kommentare zu bestehenden Events hinzugefügt werden.
Diese Aktualisierungen werden für alle Teilnehmenden synchronisiert.
Jedoch kann mit dem gesammelten Wissen aus einem Miro-Board kein weiteres Wissen automatisiert abgeleitet werden.
Die Arbeitsabläufe sind dargestellt und somit dokumentiert, dennoch ist es nicht möglich die gesammelten Daten
weiterzuverarbeiten.
Zudem ist es nicht möglich Mockups zu erstellen, hierfür bedarf es die Nutzung weiterer Tools.


\section{Aufbau der Arbeit}\label{sec:aufbau-der-arbeit}
In Kapitel~\ref{ch:grundlagen} werden zuerst die Grundlagen des~\ac{ES} und die verwendeten Technologien
zur Erstellung der zuvor erwähnten Anwendungen erläutert.
Hierbei sind die Technologien entsprechend der zugehörigen Funktion zugeordnet.
Anschließend befasst sich Kapitel~\ref{ch:implementierung} mit der Implementierung der Anwendungen.
Dabei wird genauer auf die in Kapitel~\ref{ch:grundlagen} erläuterten Technologien eingegangen.
Um den Nutzen des erstellten Web-Editors zu prüfen, wird in Kapitel~\ref{ch:evaluation} eine Evaluierung mittels Expertengespräch durchgeführt.
Die zuvor aufgestellten Ziele sollen somit, gemeinsam mit einem Experten, als erfüllt oder nicht erfüllt evaluiert werden.
Im folgenden und letzten Kapitel~\ref{ch:fazit} wird, unter Einbezug der in der Evaluation erlangten Erkenntnisse, ein Fazit zu den Zielen erstellt.
Zusätzlich soll die Umsetzung der zuvor festgelegten Funktionen anhand der Implementierung überprüft und ein Ausblick zur weiteren Entwicklung gegeben werden.
Unter anderem wird auch der mögliche Nutzen in der Lehre angesprochen.
