\chapter{Fazit}\label{ch:fazit}

Heroku ist Hure.
Das mit backend auf heroku ist bei der durchführung des Expertengesprächs während jeder Generierung in das Problem gelaufen,
dass der gebotene Speicher überschritten wurde.
Dadurch musste das Backend verhäuft neu gestartet werden, da die erstellten Mockups und Boards halb oder sogar gar nicht generiert wurde.
Einige Funktionalitäten gingen dadurch verloren.
Grund für den zu hohen Verbrauch des Speichers ist die Verwendung von Graphviz zur Generierung der UML-Diagramme.
Dafür wurde von Spring Boot eine javascript engine initialisiert.
Das zusammenspiel von Spring Boot und der gestarteten javascript engine war somit zu viel.
Dieses Problem wurde gelöst, indem das Backend in ein eigenes Docker Image verpackt wurde, bei dem Graphviz explizit vorinstalliert wurde,
um das Starten der javascript engine von spring boot zu umgehen.
Dadurch musste die Art und Weise wie das Backend auf Heroku bereitgestellt wird geändert werden.\footnote{https://devcenter.heroku.com/articles/container-registry-and-runtime}
Heroku bietet glücklicherweise die möglichkeit docker images in einer heroku eigenen registry zu hinterlegen und diese für eine app auszuführen.
Damit funktioniert die Generierung beziehungsweise das Backend wieder und steht zum aktuellen Zeitpunkt wieder in seiner ganzen Funktionalität bereit.

\todo{Machen die Erweiterungen Sinn?}

\todo{Kann der Editor von Leuten verwendet werden, die nur wenig Programmiererfahrung haben?}

\todo{Füllt diese Anwendung eine bestehende Lücke im RE?}

\todo{Wurde das zuvor gesetzte Ziel erreicht?}

\todo{Falls nein, kann auf der Grundlage dieser Arbeit etwas Besseres geschaffen werden?}

\todo{Warum ist die Anwendung kein ersatz für sowas wie Miro oder vorort Arbeit?}
Durch den Web-Editor könnte immer nur eine Person am Board arbeiten, das tritt mehrere Grundlagen des ES mit füßen.
Alle sollen mitmachen, alle sollen frei post its aufhängen können.
