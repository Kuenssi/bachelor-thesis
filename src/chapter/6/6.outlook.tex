\chapter{Ausblick}\label{ch:ausblick}

Die im Web-Editor vorhandenen Beispiele wurden in dem Editor selbst beschrieben.
Bei der Arbeit mit der Anwendung sind mehrere Feature-Anfragen aufgekommen.
Die Autovervollständigung im Codeeditor ist bisher nicht kontext abhängig.
Es werden immer alle möglichen Schlüsselwörter angeboten, auch wenn diese nach dem JSON-Schema nicht valide sind.
Hier erhält man erst bei dem Beginn der Generierung und der damit einhergehenden Validierung der Beschreibung ein Feedback.
Um die Arbeit mit dem Tool für den Entwickelnden zu erleichtern, sollte diese Autovervollständigung verbessert werden und abhängig vom Kontext
sinnvolle Vorschläge machen.
Das eben erwähnte Feedback wird im Fehlerfall als Toast angezeigt, hierbei hat die Fehlermeldung eine bestimmte Form, welche nicht in jedem Fall aussagekräftig ist.
Zudem wird zwar angezeigt, welches Element falsch ist, diese Stelle wird aber weder farblich hervorgehoben noch fokussiert.

Im Gespräch mit dem Experten aus Kapitel~\ref{sec:expertengespraech} entstanden ebenfalls weitere Funktionen, welche für die Verwendung des Tools in der Praxis von Vorteil wären.
\todo{Das aus dem Video rausholen (Übersicht -> Mehr Viewelemente -> Bild hochladen für Personalisierung(Branding) -> APis ansprechen und daten darstellen?)}

Wie bereits in Kapitel~\ref{subsec:erweiterung} angedeutet wurde, soll das in dieser Arbeit erstellte Tool auch einen Einsatz in der Lehre bekommen.
In der Veranstaltung \textit{Programmieren und Modellieren} wird den Studierenden die Methodik der objektorientierten Programmierung beigebracht.
Hierfür wurde bereits in vorherigen Semestern zur Generierung von Datenmodellen in Java die Bibliothek \textit{fulib} verwendet.
Doch auch fulibScenarios, welches die Datenmodellgenerierung über eine natürliche Sprache übernimmt, wurde als eine mögliche Alternative ausprobiert.
fulibWorkflows soll ebenfalls als eine mögliche Alternative zur Fulib-Notation ausprobiert werden.
Eine weitere Veranstaltung ist \textit{Microservices}, bei welche Studierende eine Einführung in die Web-Entwicklung und die Verwendung von
Microservices erhalten.
Da es sich dabei um separate System handelt, welche miteinander kommunizieren, um Daten zu übertragen, ist die Architektur schwierig greifbar.
Hierfür soll die Verwendung des mittels fulibWorkflows generierten Event Storming Boards als Unterstützung zum Verständnis der Abläufe dienen.

Studierende haben bereits in vergangenen Instanzen der Veranstaltung~\textit{Programmieren und Modellieren} die Web-Anwendung

\-\hspace{5cm}\url{https://fulib.org/}

kennengelernt.
Dort ist es möglich auf fulib basierende Anwendungen zu verwenden.
Neben einer natürlichsprachigen Beschreibung eines Datenmodells besteht ebenfalls die Möglichkeit ein vorgefertigtes Gradle-Projekt zu exportieren.
Da auch fulibWorkflows ein Teil der \textit{Fujaba Tool Suite} ist, soll der in dieser Arbeit erstellte Web-Editor in fulib.org integriert werden.
Somit werden Anwendungen, welche zur \textit{Fujaba Tool Suite} gehören, an einem Ort platziert und verwendet werden.
Da der Web-Editor für fulibWorkflows und fulibScenarios Objekt- und Klassendiagramme generieren und diese Anzeigen, liegt ein Zusammenlegen nahe.
Dies setzt voraus, dass fulibWorkflows ebenfalls die Generierung eines Klassenmodells mittels fulib in einem Gradle-Projekt bietet.
Diese Funktion soll ebenfalls bereitgestellt werden, um den zuvor erwähnten Einsatz in der Lehre zu ermöglichen.
