\chapter{Ausblick}\label{ch:ausblick}

Im Gespräch mit dem Experten aus Kapitel~\ref{sec:expertengespraech} entstanden ebenfalls weitere Funktionen, welche für die Verwendung des Tools in der Praxis von Vorteil wären.
Eine Ausweitung der verwendbaren Elemente in den Mockups ist eine dieser Funktionen\footnote{Siehe~\ref{fig:rec} Expertengespräch ab Minute  49:45}.
Dabei wurde angesprochen, dass es sich dabei um Standardelemente wie Checkboxen, Dropdowns und Datepicker handelt.
Es wurde auch das Erstellen von Tabellen angesprochen, allerdings bedarf es hierbei einer Abwägung, ob es möglich ist eine Tabelle in der YAML-syntax zu definieren,
welche nicht zu kompliziert für die Nutzenden ist und damit zu zeitaufwändig in einer \ac{ES}-Session wäre.
Neben diesen Elementen könne es in Betracht gezogen werden, URLs für Bilder oder sonstige Daten in Pages mit anzugeben und
daraus bei der Generierung dynamisch Bilder in den HTML-Mockups hinzuzufügen.
Diese Bilder könnten als Platzhalter oder zum Branding eines Mockups verwendet werden.
Alternativ zu einer URL ist eine weitere Überlegung, dass man Bilder hochladen kann und diese dann anstelle einer URL hinterlegt\footnote{Siehe~\ref{fig:rec} Expertengespräch ab Minute  40:50}.
Eine URL könnte auch eine API ansprechen, um Daten eines anderen Services zu erhalten.
Weiterhin wünschte sich der Experte, dass es möglich ist Kommentare zu Notes hinzuzufügen\footnote{Siehe~\ref{fig:rec} Expertengespräch ab Minute  48:02}.
Dies ist bei dem Online-Tool Miro, welches in Kapitel~\ref{sec:aufbau-der-arbeit} erwähnt wurde, bereits möglich und wurde auch nach einer Session als Möglichkeit
zum Verfeinern eines \ac{ES}-Boards genutzt\footnote{Siehe~\ref{fig:rec} Expertengespräch ab Minute  39:47}.

Im Fazit wurde bereits klargestellt, dass das erstellte Online-Tool momentan kein Ersatz für Tools wie Miro ist.
Um dies zu ändern, ist es nötig, dass jeder Teilnehmer die Möglichkeit hat Events zu einer Workflow-Beschreibung hinzuzufügen.
Eine Umsetzung dieser Funktion benötigt jedoch weitreichende neue Technologien.
Zwischen den Teilnehmern einer Session müssen die Änderungen im Editor synchronisiert werden, dafür muss zuvor die Funktion
implementiert werden, dass es so etwas wie Sessions gibt, bei denen neue Personen über einen Link oder ein Token eingeladen werden können.
Bevor diese Funktion umgesetzt wird, muss evaluiert werden, ob die erstellte Beschreibungssprache intuitiv und nutzbar ist.
Zudem ist festzustellen, ob ein Bedarf für ein solches Tool existiert, oder die bisher erstellte Anwendung zur Unterstützung eines Event Stormings genügt.

Wie bereits in Kapitel~\ref{subsec:erweiterung} angedeutet wurde, soll das in dieser Arbeit erstellte Tool auch einen Einsatz in der Lehre bekommen.
In der Veranstaltung \textit{Programmieren und Modellieren} wird den Studierenden die Methodik der objektorientierten Programmierung beigebracht.
Hierfür wurde bereits in vorherigen Semestern zur Generierung von Datenmodellen in Java die Bibliothek \textit{fulib} verwendet.
Doch auch fulibScenarios, welches die Datenmodellgenerierung über eine natürliche Sprache übernimmt, wurde als eine mögliche Alternative ausprobiert.
Ebenso soll fulibWorkflows als eine Alternative zur Fulib-Notation ausprobiert werden.
Eine weitere Lehrveranstaltung ist~\textit{Microservices}, bei welcher Studierende eine Einführung in die Web-Entwicklung und die Verwendung von
Microservices erhalten.
Da es sich dabei um separate Systeme handelt, welche miteinander kommunizieren, um Daten zu übertragen, ist die Architektur schwierig greifbar.
Hierfür soll die Verwendung des mittels fulibWorkflows generierten \ac{ES}-Boards als Unterstützung zum Verständnis der Abläufe dienen.

Studierende haben bereits in vergangenen Instanzen der Veranstaltung~\textit{Programmieren und Modellieren} die Web-Anwendung \textit{fulib.org} kennengelernt.
Dort ist es möglich auf fulib basierende Anwendungen zu verwenden.
Neben einer natürlichsprachigen Beschreibung eines Datenmodells besteht ebenfalls die Möglichkeit ein vorgefertigtes Gradle-Projekt zu exportieren.
Da auch fulibWorkflows ein Teil der \textit{Fujaba Tool Suite} ist, soll der in dieser Arbeit erstellte Web-Editor in fulib.org integriert werden.
Somit werden Anwendungen, welche zur \textit{Fujaba Tool Suite} gehören, an einem Ort platziert.
Da der Web-Editor für fulibWorkflows und fulibScenarios Objekt- und Klassendiagramme generieren und anzeigen, liegt ein Zusammenlegen nahe.
Dies setzt voraus, dass fulibWorkflows ebenfalls die Generierung eines Klassenmodells mittels fulib in einem Gradle-Projekt bietet.
Diese Funktion soll ebenfalls bereitgestellt werden, um den zuvor erwähnten Einsatz in der Lehre zu ermöglichen.
