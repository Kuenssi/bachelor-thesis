\chapter{Grundlagen}\label{ch:technologien}
\section{Event Storming}\label{sec:event-storming}
Was ist das, warum ist das cool dies das Kontrabass.

\subsection{Allgemein}\label{subsec:allgemein}
Beschreiben was die Grundideen von Brandolini sind.
Wo spielt da das DDD (Domain Driven Design) rein?

\subsection{Erweiterung}\label{subsec:erweiterung}
Was haben Albert und ich uns dazu gedacht um es nicht nur als Tool für die Wirtschaft, sondern
auch in der Lehre verwenden zu können.
Auch hier könnte man schon mit hereinbringen, dass Albert das ganze als Ablaufsbeschreibung in
Microservices verwendet hat -> Workflow raussuchen und mit youtube video belegen.

\section{Technologien}\label{sec:technologien}

\subsection{fulibWorkflows}\label{subsec:fulibworkflows}
Jaaaaa, das ist die Java Library.
Ganz geiles Teil eigentlich jetzt wo der Code auch lesbar ist.

\subsubsection{ANTLR4}\label{subsubsec:antlr4}
Geiles Ding.
Aber um Albert noch mal zu zitieren: ``ANTLR ist so zickig wie `ne Bitch im Jungle Camp''.

\subsubsection{JSON-Schema}\label{subsubsec:json-schema}
Ziemlich geil um ehrlich zu sein.
Man kann auch yaml damit definieren.

Erlaubt es die Eingaben des Nutzers zu beschränken und hints zu geben, was man machen darf
und was nicht.
Bereitstellen geht easy über schemastore.org.
Wird dabei auch von vielen IDEs benutzt.
Intellij direkt out-of-the-box, VSCode auch nachdem man die YAML Redhat Extension installiert hat.

\subsubsection{fulibTools}
Dank fulibTools ist auch fulib mit drin.
FulibTools ist zur Generierung von Objektidiagrammen genutzt worden.
Fulib (Bei FulibTools mit integriert) zur Generierung von Klassendiagrammen.

\subsection{fulibWorkflows Web-Editor FE}\label{subsec:fulibworkflows-web-editor}
Da brauchte es etwas mehr als beim BE\@.
Die Entscheidungen für die verwendeten Technologien im Frontend wurden aufgrund
der Idee der Integrierung vom Editor auf fulib.org getroffen.

\subsubsection{Angular}
Wir kennen es.
Wir lieben es.

\subsubsection{Bootstrap}
Alles für den Dackel, alles für den Club unser Leben für die schön gestylten FEs.
Simpel, oder?
Ja, okay.
Man nutzt dann auch ng-bootstrap für Angular Anwendungen.
Natürlich auch noch bootstrap-icons.
Will ja niemand traurig machen und von Adrian verdroschen werden.

\subsubsection{Codemirror}
Schönes Ding.
Ngx-codemirror ist es dann speziell für eine Angular Anwendung geworden.
Eigentlich alles out of the box benutzt.

\subsubsection{Angular-split}
Find ich schon gut zu erwähnen.
Ohne die geile dependency wäre das FE nie so pornös geworden.

\subsubsection{file-saver}
Weitere erwähnenswerte dependency.
Wird genutzt um Dateien, die vom BE kommen, auch herunterladen zu können.

\subsection{fulibWorkflows Web-Editor BE}\label{subsubsec:backend}
Yeey alle Technologien die ich im Backend benutzt habe.

\subsubsection{Spring Boot}
Framework mit dem man easy mal ein backend generiert bekommt.
Durch Java und viele Dependency allerdings alles andere als ein leichtgewicht.

Dennoch musste ein Java backend her, da sonst fulibWorkflows nicht hätte integriert werden können.
Jedenfalls nicht ohne noch mehr middle ware.

Zudem hatte ich im Praktikum mit Spring Boot erfahrungen gesammelt.
Die Verwendung von Annotations und dem aufsplitten zwischen Controller und Service ist mir bereits
durch Nest.js bekannt gewesen.

Dennoch muss man sagen, dass durch die von Spring Boot bereits integrierten libraries nichts weiter
außer fulibWorkflows hinzugefügt werden musste.
Immerhin umfasst der Code vom Backend vielleicht 300 Lines of Code.

\subsection{Deployment}\label{subsec:deployment}
Ein Web Editor will natürlich für alle erreichbar sein.
Und fulibWorkflows muss auch irgendwo bereitgestellt werden, damit es das Backend und alle anderen
interessierten benutzen können.

\subsubsection{MavenCentral}\label{subsubsec:mavencentral}
MavenCentral ein wirklicher Hussarones was das publishen angeht.
Glücklicherweise ist fulibWorkflows Teil der Fujaba Tool Suite, wodurch die benötigten
Zugriffsrechte bereits vorhanden und andere Libraries bereits gepublished wurden.
Hierdurch war es recht schnell möglich mit dem zuvor erworbenem Wissen fulibWorkflows
zu publishen.

\subsubsection{Heroku}\label{subsubsec:heroku}
Der Web-Editor soll immer erreichbar sein.
Dies ist durch Heroku nur bedingt möglich.
Heroku bietet allerlei Möglichkeiten verschiedenste Anwendungen bereitzustellen.
Auch mit einem kostenlosen Plan ist es ohne Probleme möglich solch kleine Anwendungen bereitzustellen.

FE Deployment war easy, auch wenn ich erstmal wieder in eins meiner früheren Projekte gucken musste.
BE Deployment war kniffliger, doch man ist nie der erste der eine Spring Boot application
auf Heroku deployen will.
Daher Tutorial reingefahren und ab ging der gebutterte Lachs.